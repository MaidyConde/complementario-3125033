\documentclass{IEEEoj-data}
\usepackage{cite}
\usepackage{amsmath,amssymb,amsfonts}
\usepackage{algorithmic}
\usepackage{graphicx,color}
\usepackage{textcomp}
\usepackage{hyperref}
\usepackage{balance}
\def\BibTeX{{\rm B\kern-.05em{\sc i\kern-.025em b}\kern-.08em
    T\kern-.1667em\lower.7ex\hbox{E}\kern-.125emX}}
\AtBeginDocument{\definecolor{ojcolor}{cmyk}{0.93,0.59,0.15,0.02}}

\begin{document}
%\receiveddate{00 April, 2024}
%\reviseddate{11 April, 2024}
%\accepteddate{00 April, 2024}
%\publisheddate{00 May, 2024}
%\currentdate{24 June, 2024}
%\doiinfo{DD.2024.0910000}

\title{\textcolor{black}{Descripción:} \textcolor{ieeedata}{\textit{TuTurismo Neiva: Fomentando la Cultura y el Turismo a Través de la Innovación Tecnológica}}}

\author{Valentina Silva Garrido, Mariana González Calderón, Maydy Viviana Conde Ladino, Dylan Santiago Narvaez Pinto, Isabela Gutierrez Cordoba, Ivan Andres Murcia Epia}
\affil{Centro de la industria la empresa y los servicios,  Neiva, Colombia}

\corresp{AUTOR CORRESPONDIENTE: Maydy Viviana Conde Ladino(e-mail: maidy_conde@soy.sena.edu.co).}

\authornote{Autores contribuyeron igualmente a este artículo.}

\maketitle

\section*{RESUMEN} 

Este artículo describe la aplicación TuTurismo Neiva, diseñada para promover el turismo en la ciudad de Neiva a través de una plataforma digital interactiva. La aplicación permite a los usuarios explorar, descubrir y calificar los sitios y monumentos turísticos de la ciudad, facilitando la integración de la historia local con modernas herramientas de evaluación y recomendación. La metodología utilizada en el desarrollo de la aplicación incluye enfoques ágiles como Scrum y XP, además de herramientas gráficas como Canva para la creación de interfaces. Los resultados muestran que la aplicación ha mejorado la visibilidad de sitios menos conocidos y ha favorecido la interacción de los usuarios con el patrimonio cultural de Neiva. Además, la plataforma incluye un sistema PQRSFD, que permite a los usuarios expresar opiniones y sugerencias, contribuyendo a un proceso de mejora continua. En conclusión, TuTurismo Neiva no sólo fomenta el conocimiento cultural sino que también promueve el desarrollo económico y social de la ciudad, brindando una experiencia interactiva para los turistas.

\begin{IEEEkeywords}
Mobile application, Tourism, Culture, Neiva, Innovation, History, PQRSFD
\end{IEEEkeywords}

\section*{1. Marco Teórico }

El concepto de turismo digital ha ganado relevancia con el avance de la tecnología móvil y la conectividad online, permitiendo a los usuarios acceder a información en tiempo real sobre los destinos turísticos. Según Dredge (2006), las plataformas digitales promueven la interacción entre turistas y locales, lo que enriquece la experiencia cultural y promueve la sostenibilidad. De manera similar, los sistemas de retroalimentación como PQRSFD (Preguntas, Quejas, Reclamos, Sugerencias, Felicitaciones y Denuncias) han sido identificados como cruciales para mejorar la calidad de los servicios turísticos y la satisfacción del cliente (Fitzsimmons, 2014).  Además, estudios previos como el de O'Connor (2010) destacan cómo las herramientas tecnológicas pueden transformar la industria del turismo al proporcionar plataformas de fácil acceso para la promoción de destinos.

\section*{2. Metodología }

El desarrollo de TuTurismo Neiva siguió un enfoque ágil utilizando las metodologías Scrum y XP para asegurar un proceso iterativo y flexible en la construcción de la aplicación. Scrum permitió dividir el proyecto en sprints, cada uno enfocado en una funcionalidad específica como geolocalización de sitios turísticos, integración del sistema de evaluación de usuarios y creación del sistema PQRSFD. Para el diseño visual se utilizó la plataforma Canva, lo que facilitó la creación de interfaces intuitivas y visualmente atractivas. Se realizaron pruebas de usabilidad con un grupo de usuarios locales y turistas, cuyos comentarios fueron fundamentales para ajustar la experiencia y funcionalidad de la aplicación.

\section*{3. Resultados }

TuTurismo Neiva ha logrado compilar una importante base de datos de sitios turísticos de Neiva, con más de 50 lugares registrados y evaluados por los usuarios. A través del sistema de calificación, los turistas pueden valorar estos sitios, lo que ha ayudado a resaltar atracciones menos conocidas. Además, el sistema PQRSFD ha permitido recopilar información valiosa sobre las experiencias de los usuarios, contribuyendo a la mejora continua de la aplicación. Los gráficos y tablas que se presentan a continuación muestran el crecimiento en el número de usuarios activos y la distribución geográfica de los sitios más visitados. (Incluya aquí gráficos y tablas: imágenes con datos de usuarios, calificaciones del sitio y distribución geográfica de las visitas).

\section*{4. Discusión }

Los resultados obtenidos en este estudio sugieren que TuTurismo Neiva ha logrado su objetivo de incrementar la visibilidad y accesibilidad de los sitios turísticos de la ciudad. Sin embargo, se ha observado que los usuarios todavía prefieren los sitios más conocidos, lo que destaca la necesidad de mejorar las estrategias de marketing y promoción digital para los lugares menos visitados. En comparación con estudios anteriores sobre plataformas turísticas digitales, la integración de un sistema de retroalimentación como PQRSFD en la aplicación ha demostrado ser una práctica eficaz para promover la interacción del usuario y mejorar los servicios. Las limitaciones incluyen la falta de recursos para una promoción nacional más amplia de la aplicación y la necesidad de realizar más pruebas a gran escala.

\section*{5. Conclusiones}

En conclusión, TuTurismo Neiva ha logrado crear una plataforma efectiva para promover el turismo en Neiva, mejorando la interacción del turista con la cultura local. La implementación de la metodología ágil, combinada con el uso de herramientas de diseño intuitivas, ha permitido la creación de una aplicación funcional y de fácil acceso. Se recomienda expandir la aplicación a otras ciudades y realizar una campaña de marketing digital más agresiva para aumentar el número de usuarios y la visibilidad de los sitios menos conocidos. Además, se sugiere incorporar más funcionalidades, como itinerarios personalizados y una plataforma para que los usuarios creen sus propios recorridos turísticos.

\end{document}