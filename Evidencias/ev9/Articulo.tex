% Archivo: personal_overview.tex
\documentclass{IEEEoj-data}
\usepackage{cite}
\usepackage{amsmath,amssymb,amsfonts}
\usepackage{algorithmic}
\usepackage{graphicx,color}
\usepackage{textcomp}
\usepackage{hyperref}
\usepackage{balance}
\def\BibTeX{{\rm B\kern-.05em{\sc i\kern-.025em b}\kern-.08em
    T\kern-.1667em\lower.7ex\hbox{E}\kern-.125emX}}
\AtBeginDocument{\definecolor{ojcolor}{cmyk}{0.93,0.59,0.15,0.02}}

\begin{document}
\title{\textcolor{black}{Descripcion:} \textcolor{ieeedata}{\textit{Una descripción personal y profesional}}}

\author{Maydy Viviana Conde Ladino\authorrefmark{1}, (Student, Software Development)}
\affil{Tecn\'ologo en An\'alisis y Desarrollo de Software, Formaci\'on en Soluciones Tecnol\'ogicas, Neiva, Colombia}
\corresp{CORRESPONDING AUTHOR: Maydy Viviana Conde Ladino (e-mail: contacto@example.com).}
\authornote{El autor contribuyó igualmente a este artículo. Este trabajo es una descripción personal con fines educativos.}
\markboth{IEEE-DATA Descriptor Article Template}{Author \textit{et al.}}

\begin{abstract}
Soy estudiante del Tecn\'ologo en An\'alisis y Desarrollo de Software, con un enfoque en la creaci\'on y mantenimiento de soluciones tecnol\'ogicas. He desarrollado habilidades t\'ecnicas en programaci\'on, destacando en lenguajes como Java y JavaScript, con un s\'olido enfoque en el desarrollo backend. Me considero una persona responsable, comprometida y adaptable, capaz de trabajar de manera eficiente tanto de forma independiente como en equipo.\\

He participado en diversos proyectos peque\~nos, donde he aplicado mis conocimientos para dise\~nar e implementar soluciones software escalables y de alta calidad. Adem\'as, disfruto dedicar mi tiempo libre a jugar voleibol y leer, actividades que fomentan el trabajo en equipo, la concentraci\'on y la creatividad.
\end{abstract}

\begin{IEEEkeywords}
Java, JavaScript, desarrollo backend, ingeniería de software, trabajo en equipo, voleibol,
lectura
\end{IEEEkeywords}

\maketitle

\section*{INFORMACIÓN}

Maydy Viviana Conde Ladino, de 18 a\~nos, es una estudiante de software comprometida con la excelencia t\'ecnica. Su enfoque acad\'emico y profesional est\'a orientado hacia la creaci\'on de soluciones tecnol\'ogicas, utilizando principalmente lenguajes como Java y JavaScript. A trav\'es de diversos proyectos acad\'emicos y personales, ha perfeccionado su habilidad para desarrollar soluciones escalables y de alta calidad. Sus intereses en el voleibol y la lectura refuerzan su compromiso con el trabajo en equipo y la creatividad, habilidades esenciales en el desarrollo de software.

\section*{MÉTODOS DE COLECCIÓN Y DISEÑO}

El contenido presentado se recopil\'o a trav\'es de experiencias acad\'emicas y proyectos pr\'acticos desarrollados en el marco del Tecn\'ologo en An\'alisis y Desarrollo de Software. Cada proyecto incluy\'o la identificaci\'on de problemas, dise\~no de soluciones y programaci\'on en entornos como Java y JavaScript. Estas experiencias fueron documentadas y adaptadas para mostrar un perfil profesional claro y conciso.

\section*{INTRODUCCI\'ON}

La arquitectura de software es una disciplina fundamental para el desarrollo de sistemas complejos y escalables. Este art\'iculo examina aspectos clave, abordando temas como la toma de decisiones arquitect\'onicas y su impacto en el dise\~no de sistemas, los patrones de dise\~no que facilitan soluciones reutilizables y efectivas, y la evoluci\'on de los microservicios como modelo arquitect\'onico.

\section*{TOMA DE DECISIONES EN LA ARQUITECTURA DE SOFTWARE}

La toma de decisiones en la arquitectura de software es un proceso cr\'itico que influye directamente en el \u00e9xito de un proyecto. Las decisiones iniciales pueden definir la escalabilidad, mantenibilidad y eficiencia del sistema. Este apartado explora la racionalidad detr\'as de estas decisiones y los sesgos que pueden influir en los arquitectos, proponiendo estrategias para optimizar el proceso de dise\~no.

\section*{PATRONES DE DISE\~NO EN EL DESARROLLO DE SOFTWARE}

Los patrones de dise\~no ofrecen soluciones estandarizadas a problemas comunes en el desarrollo de software. Ejemplos como el Patr\'on de Observador, Decorador y F\'abrica Abstracta ilustran c\'omo estos modelos facilitan el desarrollo de software eficiente y robusto. Adem\'as, promueven una comunicaci\'on efectiva entre equipos y reducen la probabilidad de errores costosos.

\section*{INTRODUCCI\'ON A LA ARQUITECTURA DE SOFTWARE}

La arquitectura de software proporciona una estructura organizativa que mejora la comprensi\'on de sistemas complejos. Este apartado aborda estilos arquitect\'onicos comunes y su aplicaci\'on en escenarios pr\'acticos, utilizando estudios de caso que destacan los beneficios de una buena representaci\'on arquitect\'onica.

\section*{CALIDAD EN LA ARQUITECTURA DE MICROSERVICIOS}

La arquitectura de microservicios (MSA) ha transformado la forma en que se desarrollan y operan los sistemas. Este estudio revisa los atributos de calidad clave en MSA, como la flexibilidad, escalabilidad y eficiencia, destacando herramientas y estrategias que facilitan su implementaci\'on efectiva en entornos DevOps.

\section*{INTRODUCCI\'ON A LAS ARQUITECTURAS DE MICROSERVICIOS}

Los microservicios representan una evoluci\'on respecto a las arquitecturas monol\'iticas tradicionales. Este apartado presenta los principios b\'asicos de los microservicios, su comparaci\'on con los modelos monol\'iticos

\section*{CONCLUSIÓN DEL ARTÍCULO SOBRE ARQUITECTURA DE SOFTWARE}

La arquitectura de software es un pilar fundamental en el desarrollo de sistemas modernos. Las decisiones tomadas en la fase de diseño arquitectónico tienen un impacto directo en la calidad, escalabilidad y mantenibilidad de un software.Este artículo ha explorado diversos aspectos clave de la arquitectura de software, desde la importancia de la toma de decisiones informadas hasta la evolución hacia modelos más modernos como los microservicios.

\end{document}






